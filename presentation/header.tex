\documentclass[11pt]{beamer}





\usepackage{lmodern} 		% Diese beiden packages sorgen für echte 
\usepackage[T1]{fontenc}	% Umlaute.

\usepackage{amssymb, amsmath, color, graphicx, float, setspace, tipa}
\usepackage[utf8]{inputenc} 
\usepackage[english]{babel}
\usepackage[justification=centering]{caption}
\addto\captionsenglish{\renewcommand{\figurename}{}} %Abbildungen nicht bzw. anders beschriften.


%\usepackage[pdfpagelabels,pdfstartview = FitH,bookmarksopen = true,bookmarksnumbered = true,linkcolor = black,plainpages = false,hypertexnames = false,citecolor = black, breaklinks]{hyperref}
%\usepackage{url}
\usepackage{longtable} 		%Seitenübergreifende Tabelle. Vorlage siehe unten
\newtheorem*{bem}{Bemerkung} % Neue Theorem-Umgebung: Bemerkung
\newcommand{\fillframe}{\vskip0pt plus 1filll} 



%-----------------
%BEAMER-SPEZIFISCH
%-----------------

\usetheme{default}
% Verschiedene Varianten von usetheme, usecolortheme und usefonttheme kann man hier ausprobieren: http://deic.uab.es/~iblanes/beamer_gallery/

% \usetheme{
% 	AnnArbor | Antibes | Bergen |
% 	Berkeley | Berlin | Boadilla |
% 	boxes | CambridgeUS | Copenhagen |
% 	Darmstadt | default | Dresden |
% 	Frankfurt | Goettingen |Hannover |
% 	Ilmenau | JuanLesPins | Luebeck |
% 	Madrid | Malmoe | Marburg |
% 	Montpellier | PaloAlto | Pittsburgh |
% 	Rochester | Singapore | Szeged |
% 	Warsaw
% }
%Interessant scheinen: Boadilla, boxes, CambridgeUS, default, (Goettingen), Hannover, Madrid, Montpellier, Pittsburgh, Rochester, Singapore, Szeged, 

\usecolortheme{dove}
% \usecolortheme{
% 	albatross | beaver | beetle |
% 	crane | default | dolphin |
% 	dove | fly | lily | orchid |
% 	rose |seagull | seahorse |
% 	sidebartab | structure |
% 	whale | wolverine
% }

\usefonttheme{structurebold}
% 	default | professionalfonts | serif |
% 	structurebold | structureitalicserif |
% 	structuresmallcapsserif
% }


%\useinnertheme{
% 	circles | default | inmargin |
% 	rectangles | rounded
% } Am besten sein lassen.


% \useoutertheme{
% 	default | infolines | miniframes |
% 	shadow | sidebar | smoothbars |
% 	smoothtree | split | tree
% } Am besten sein lassen.



\setbeamercovered{transparent} %Halbtransparente Overlays (was als nächstes Element auf der Folie gezeigt wird)
\beamertemplatenavigationsymbolsempty % Entfernt Navigationssymbole unten
%\setbeamertemplate{footline}[frame]  % Seitenzahlen
    \setbeamertemplate{footline}{%
    	\raisebox{5pt}{\makebox[\paperwidth]{\hfill\makebox[10pt]{\hyperlink{tableofcontents}{\scriptsize\insertframenumber}}}}}



%---------------------
%--Metainformationen--
%---------------------
\title{Halo and Sub-Halo Finding in Cosmological N-body Simulations}

\author[M. Ivkovic]{
	Mladen Ivkovic
}
\institute[ICS -- UZH]{Institute for Computational Science\\ University of Zurich}

\date[08.12.2017]{08. June 2017}


% \title[Kurzform]{Vortrag zur Berechenbarkeit}
%     Titel des Vortrages
% \subtitle[Kurzform]{Untertitel}
%     Untertitel
% \author[M. Schulz]{Michael Schulz}
%     Autor festlegen
% \institute[IfI -- HU Berlin]{Institut für Informatik\\ Humboldt-Universität zu Berlin}
%     Angabe des Institutes
% \date[26.05.06]{26. Mai 2006}
%     Datum der Präsentation, alternativ kann mittels \date{\today} auch das aktuelle Datum eingetragen werden.
% \logo{\pgfimage[width=2cm,height=2cm]{hulogo}}
%     Die Datei hulogo.pdf (bzw. hulogo.png, hulogo.jpg, hulogo.mps bei Verwendung von pdftex als Backend) als Logo auf allen Folien, hier mithilfe des Paketes pgf.
% \titlegraphic{\includegraphics[width=2cm,height=2cm]{hulogo}}
%     Die Datei hulogo.pdf (bzw. analog wie bei \logo auch entsprechendes Format) als Logo nur auf der Titelseite unter Verwendung des Paketes graphicx.




%-------------------------------
% Neue Befehle
%-------------------------------

\newcommand{\del}{\partial}
%differential d
\newcommand{\de}{\mathrm{d}}

%\newcommand{\ramses}{\textsc{ramses}}
%\newcommand{\phew}{\textsc{phew}}
%\newcommand{\dice}{\textsc{dice}}
\newcommand{\ramses}{\texttt{RAMSES}}
\newcommand{\phew}{\texttt{PHEW}}
\newcommand{\dice}{\texttt{DICE}}

\newcommand{\dt}{\texttt{dice-twobody}}
\newcommand{\ds}{\texttt{dice-levels}}
\newcommand{\cosmo}{\texttt{cosmo}}
\newcommand{\simple}{simple unbinding}
\newcommand{\neigh}{accounting for neighbours}
\newcommand{\iter}{iterative properties determination}
\newcommand{\phewon}{PHEW only}





%===================
% BIBLIOGRAPHY
%===================
\usepackage[backend=bibtex,sorting=nyt,bibencoding=ascii,citestyle=authoryear]{biblatex}
\addbibresource{references.bib}
\usepackage{csquotes} %recommended when using babel and biblatex
%Abbreviations for Bibliography: http://adsabs.harvard.edu/abs_doc/aas_macros.html
\newcommand{\aap}{Astronomy and Astrophysics}
\newcommand{\mnras}{Monthly Notices of the RAS}
\newcommand{\apj}{The Astrophysical Journal}
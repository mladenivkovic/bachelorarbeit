\begin{table}[!htb]
	{\footnotesize
	\begin{centering}
		\def\arraystretch{1.2}
		\begin{tabular}[t]{l | R{1.7cm} |  R{1.7cm} |  R{1.8cm}  ||  R{1.7cm}  |  R{1.7cm}  |  R{1.8cm}  |}
			%
			%=======================
			% DICE TWO 
			%=======================
			%
			\cline{2-7}
							& \multicolumn{6}{|c|}{\textbf{\dt-dataset} ($2.4 \cdot 10^5$ particles, 21.8$\%$ of which are in subhalos) } \\
			\cline{2-7}
							& \multicolumn{3}{|c||}{Sequential execution} & \multicolumn{3}{c|}{Parallel execution} \\
			\cline{2-7}
							&\ramses\	 & \phew    &  unbinding & \ramses  & \phew\      &  unbinding\\
			\hline
%			Wall time (s)	&	16.19    &	19.85  	 &	21.82	 &	 6.52	  & 	8.93  	 &	12.08	\\[-1.5ex]
%							& +34.8 $\%$ & +9.9 $\%$ &	-		 & +85.3 $\%$ & +35.3 $\%$   &	-		\\[1.5ex]
			CPU time (s)	&	16.18	 &	19.83	 &	21.81	 &	26.06	  &	35.71		 &	48.30	\\[-1.5ex]
							& +34.8 $\%$ & +10.0 $\%$& -		 & +85.3 $\%$ & +35.3 $\%$   &  -		  \\[1.5ex]
			PMUPC (Mb)		&	390.50	 &	401.37	 &	452.37	 &	396.76	  &	551.09	     &	572.54	\\[-1ex]
							& +15.8 $\%$ & +12.7 $\%$& -		 & +44.3 $\%$ & +3.9 $\%$    &  -		\\
			\hline
			\multicolumn{7}{c}{}\\
			%
			%=======================
			% DICE SUB 
			%=======================
			%
			\cline{2-7}
							& \multicolumn{6}{|c|}{\textbf{\ds-dataset} ($1.26 \cdot 10^6$ particles, 23.4$\%$ of which are in subhalos) } \\
			\cline{2-7}
							& \multicolumn{3}{|c||}{Sequential execution} & \multicolumn{3}{c|}{Parallel execution} \\
			\cline{2-7}
							&\ramses\	 & \phew    &  unbinding & \ramses\ & \phew\      &  unbinding \\
			\hline
%			Wall time (s)	&	56.77	 &	82.32	 &	88.41	 &	70.40	  &	 90.33	     &	111.24	\\[-1.5ex]
%							& +55.7 $\%$ & +7.4 $\%$ &  -		 &+58.0 $\%$  & +23.1 $\%$   &  -   \\[1.5ex]
			CPU time (s)	&	56.77	 &	82.31	 &	88.41	 &	281.60	  &	361.30		 &	444.93	\\[-1.5ex]
							& +55.7 $\%$ & +7.4 $\%$ &	-		 &+58.0 $\%$  &	+23.1 $\%$   &	- 	\\[1.5ex]
			PMUPC (Mb)		&	1313.36	 &	1358.45	 &	1562.06	 &	405.96	  &	565.26		 &	625.64	\\[-1ex]
							& +18.9 $\%$ & +15.0 $\%$&   -		 & +54.1 $\%$ &+10.7 $\%$    &   -		\\
			\hline
			\multicolumn{7}{c}{}\\
			%
			%=======================
			% COSMO 
			%=======================
			%
			\cline{2-7}
							& \multicolumn{6}{|c|}{\textbf{\cosmo-dataset} ($128^{3}$ particles, 9.0$\%$ of which are in subhalos) } \\
			\cline{2-7}
							& \multicolumn{3}{|c||}{Sequential execution} & \multicolumn{3}{c|}{Parallel execution} \\
			\cline{2-7}
							& \ramses	& \phew    &  unbinding & \ramses\ & \phew      &  unbinding\\
			\hline
%			Wall time (s)	&	-		&	-		&	-		&	95.94	 &	108.24	    &	112.12	\\[-1.5ex]
%							&			&			&			& +16.9 $\%$ & +3.6 $\%$    & -			\\[1.5ex]
			CPU time (s)	&	-		&	-		&	-		&	383.72	 &	432.93	  	&	448.46	\\[-1.5ex]
							&   		&			&			& +16.9 $\%$ & +3.6 $\%$    & -			\\[1.5ex]
			PMUPC (Mb)		&	-		&	-		&	-		&	1335.61	 &	1342.77		&	1526.65	\\[-1ex]
							&			&			&			& +14.3 $\%$ & +13.7 $\%$   &	-		\\
			\hline
			\multicolumn{7}{c}{}
		\end{tabular}
	\end{centering}
	} %footnotesize
	\caption{%
		Performance measurements for sequential and parallel executions on the three datasets used for tests. 
		The parallel runs were executed on 4 cores. The CPU time of the parallel executions is the total CPU time of all 4 cores. ``\emph{PMUPC}'' stands for ``peak memory usage per core''.		
		``\ramses'' measurements consist of advancing the simulation for a timestep. ``\phew'' runs do the same, but include clump finding, while ``unbinding'' runs include clump finding and particle unbinding.
		The given values are the highest values out of 10 measurements. 
		The percentages below the measurements give the relative additional cost of the unbinding run.
	}%
	\label{tab:resource-usage-measurements}%
\end{table}